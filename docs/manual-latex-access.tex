\documentclass[12pt,a4paper]{report}
\usepackage[pdftex]{hyperref}
\begin{document}
\title{The LaTeX-access manual}
\author{Alastair Irving <\href{mailto:alastair.irving@sjc.ox.ac.uk}{alastair.irving@sjc.ox.ac.uk}>\\
  Robin  Williams <\href{mailto:rmw205@exeter.ac.uk}{rmw205@exeter.ac.uk}>\\
  Daniel Dalton <\href{mailto:<daniel.dalton10@gmail.com}{daniel.dalton10@gmail.com}>\\
  Nathaniel Schmidt <\href{mailto:nathanieljsch@westnet.com.au}{nathanieljsch@westnet.com.au}>\\
  Stefan Moisei <\href{mailto:vortex37@gmail.com}{vortex37@gmail.com}>}
\maketitle
\tableofcontents

Latex-access -- provides a blind person with a more efficient means of
interacting with LaTeX for mathematical and scientific documents.

\chapter{Introduction}
\label{ch-introduction}

The latex-access project is designed to provide a realtime
translation of a line of LaTeX in to braille, concentrating
on the Nemeth code, whichh can be read on a refreshable braille
display. This will greatly improve the ease of use of
LaTeX to blind mathematicians and scientists. The project also
translates the current line into english speech which is
easier to listen to than LaTeX source.

Note that this project is largely aimed at people wishing
to read LaTeX using a refreshable braille display and/or
speech synthesisor[B, and people who will probably
want to edit LaTeX documents. For example, as a
university student,
I receive my worksheets in LaTeX format, and
produce my work using LaTeX. Using the latex\_access
package, I am able to
get a fairly good translation of the question
and then an on-the-fly translation of my work as
I produce it. If you are
not concerned with editing LaTeX documents
and simply want a braille translation of an
entire laTeX document, then this
project is not for you.

There is also a very low traffic mailing list, which is worth subscribing to if you have any
queries, problems, suggestions or ideas. All current developers are
subscribed to this list and are very willing to assist. To subscribe
send an email with the word ``subscribe'' in the subject to:
\href{mailto:latex-access-devel-request@lists.sourceforge.net}{latex-access-devel-request@lists.sourceforge.net}. To
post to the list send emails to \href{mailto:latex-access-devel@lists.sourceforge.net}{latex-access-devel@lists.sourceforge.net}

\section{Purpose}
\label{subchap-purpose}
It is widely thought that LaTeX is a good system for a blind
mathematician or scientist to use to create and read
scientific documents, as it is a linear code and so the user does
not have to perceive two-dimensional concepts, such
as fractions and column vectors.
By reading this linear code, a blind person can take in and
understand scientific documents in the same way that a
sighted person would do by studying a printed document.
It should be noted that normally, laTeX is just a source from which
documents are converted in to an
attractive-looking, typeset document that can be printed or viewed
on screen, often in a .pdf, .dvi or .ps format. For
various technical reasons, documents in such formats are
currently inaccessible with current screen-reading technology.
The best current solution therefore is not to concern
ourselves with documents in these formats, but rather to
read
and interpret the LaTeX source code itself.

\section{Reading a LaTeX document}
\label{subchap-reading-latex-document}

It is entirely possible to read a LaTeX document simply by reading
the LaTeX source itself. This however, is often a
time-consuming and pain-staking process, and it is often not
particularly nice to read. For example, the LaTeX source
for the quadratic formula is\\
\mbox{\$\$x=\textbackslash frac\{-b\textbackslash pm\textbackslash sqrt\{b\^2-4ac\}\}\{2a\}\$\$}\\
It is therefore the aim of the project to translate a
line of LaTeX in to a line of Niemeth braille code, which
can be
read using a refreshable braille display. The project
also aims to provide an audible translation of the
LaTeX source
which will be output through current screen-reading
technology.          

\section{Current features}
\label{subchap-current-features}


latex-access currently contains the following features.

\begin{itemize}
\item Translation of several mathematical expressions from LaTeX to
Niemeth braille. These include, but are not confined
to:
\begin{itemize}
\item Translation of fractions, both numerical and
algebraic.
\item Translation of trigonometric
functions and hyperbolic functions.
\item Translation of powers,
including square roots.
\item Translation of
expressions used in calculus, including partial derivatives.
\item Translation of two component and three component column vectors,
  not in to Niemeth braille format but in to a row vector so that they
  can be read on a single line braille display.
\item Translation of several mathematical symbols, such as the Greek letters.
\item Many commands used to create a visually attractive document are
  either translated or ignored, often it is not necessary to see some
  formatting commands.
\end{itemize}

\item Translation of several of the above to audible speech.
\item A matrix browser feature to enable easier reading of larger
  matrices in LaTeX, see the description below.
\item Support for custom defined LaTeX commands.
\end{itemize}

\chapter{Installation}
\label{ch-installation}

\section{Obtaining the source}
\label{subchap-obtaining-source}
The package is hosted by svn at,
\url{https://latex-access.svn.sourceforge.net/svnroot/latex-access}.

This link will take you to a web interface of the svn tree, but you'll
probably want to checkout the code so you can install it. If you run
windows see section~\ref{subsubchap-windows}, and if you run linux
see section~\ref{subsubchap-linux}.

\subsection{Linux}
\label{subsubchap-linux}

Under Linux, the standard subversion command line client works
well. This can usually be installed on debian based distros by running\\
\mbox{apt-get install subversion}\\
\# (as root)

Then type\\
\mbox{svn co https://latex-access.svn.sourceforge.net/svnroot/latex-access latex-access}

This will check the package out into the directory latex-access.

In future feel free to run ``svn up'', to pull the latest updates from
the server. (from within the latex-access directory).

\subsection{Windows}
\label{subsubchap-windows}

On windows the ``Tortoise SVN'' client works well.\\
\url{http://tortoisesvn.tigris.org/}

Once you have checked out the sourcecode from\\
\url{https://latex-access.svn.sourceforge.net/svnroot/latex-access}\\
continue with the installation
process.

Note, you should periodically pull the latest updates from the server to
get the latest and greatest features of latex-access.

\section{Installing the package}

Currently latex-access interacts with three different
programs. Therefore, the installation process varies slightly.

\begin{itemize}
\item Jaws for windows -- most developed front-end to the project, and
  allows Jaws to interact with latex-access within most editors. See
  section~\ref{subsubchap-jaws}.
\item Emacs/emacspeak -- Provides both Braille and speech access when
    working with latex under the emacs editor. This means anyone using a
    Linux system can use the project. Usage under Linux is very stable,
    but there are still a few little things to be done, see the README
    in the emacs branch for details. Currently has not been tested with
    the windows emacs version. See section~\ref{subsubchap-emacs} for
    installation instructions.
\item NVDA -- provides latex-access for usage with most
      editors. Under heavy development still, but it provides many
      features of latex-access. See section~\ref{subsubchap-nvda} for
      installation instructions.
\end{itemize}

For specific notes on each individual front-end it is worth looking at the
READMES in their respective branches. You can also find details about
contacting the developer of the specific front-end from the readme.

\subsection{Jaws}
\label{subsubchap-jaws}

To install the scripts on a machine running windows and Jaws
version 5 or higher, do the following.

\begin{enumerate}
\item Download the latest stable release of version 2 of the Python software from
\url{http://www.python.org} and install it.

Note: LaTeX-access has not been tested with python 3 and probably won't work with it.
\item Download the pywin32 package  from
\url{https://sourceforge.net/projects/pywin32} ensuring you have the correct
file to match your version of python. Install this.
\item Create a directory on the c: drive named latex\_access. (The name
of the directory or it's level of depth in the filesystem does not
matter, however choosing a directory with a name easy to remember and
not too deep in the filesystem makes the next step easier.
\item Obtain the latest version of the scripts, (although you've
presumably already done this).  For details of how to obtain the
latest version please see section~\ref{subchap-obtaining-source}.
\item Extract the files and folder in the zip file to the latex\_access
folder you created earlier. To do this, either simply run the zip file
with an application such as winrar and copy everything to the folder,
or extract the files and folder to the relevant location by right
clicking (or equivalent) on the file within windows explorer.
\item Copy the files from within the jaws folder to the folder where your
jaws scripts are located. These are usually in a path such as\\
\mbox{c:\textbackslash documents and settings\textbackslash\%username\%\textbackslash application data\textbackslash freedom scientific\textbackslash jaws\textbackslash\%jaws version\%\textbackslash settings\textbackslash enu}\\
or\\
\mbox{c:\textbackslash Users\textbackslash \% username\% \textbackslash AppData\textbackslash freedom scientific\textbackslash jaws\textbackslash\%jaws version\%\textbackslash settings\textbackslash enu}\\
(This can be reached using the explore jaws submenu of the jaws menu
in your start menu).
\item Now open a command prompt by going to run, and typing ``cmd''.
Switch to the latex\_access directory by typing 'cd \%directory\%' for
example, type ``cd c:\textbackslash latex\_access''
\item Type\\
``latex\_access\_com.py''\\
you should hear ``latex\_access registered''.  Now
register the matrix processor object by typing\\
``matrix\_processor.py''
You should hear a similar message.  Exit the command
prompt by typing exit.

Note: the object here is to run the specified python files with 
python, so the above will only work if python is the default program
associated with .py files.  If it is not then try 
python latex\_access\_com.py
and if this fails then use the full path to your python installation,
for example \\
\mbox{c:\textbackslash python26\textbackslash python.exe latex\_access\_com.py}\\
then repeat with matrix\_processor.py
\item Finally, open the confignames.ini file in your jaws scripts folder
using a text editor such as notepad. After the line which reads
[ConfigNames]
enter a new line as described below.

As many different applications can be used to read or write LaTeX, the scripts have a
generic name latex.j**. Therefore, we use this file to add an alias,
so that many different applications can be used without needing to
change the name of each Jaws script to match the executable name of
the application used to write LaTeX. So for example, as a user of
winedt, I have ``winedt=latex'' If you use notepad to write and read
LaTeX, you should add ``notepad=latex'' etc.  If you are one of multiple
users on the computer, you may have to search to find the
confignames.ini file. The file will be found within the same file
structure, within the all users (or something similar) directory,
within the documents and settings directory.

Note: with older versions of jaws you could just enter the lines at
the bottom of the file, but in more recent versions the .ini file
contains more sections and the lines must be inserted in the
[ConfigNames] section.
\item Load the relevant program for writing/reading LaTeX, and press
ctrl+m to initialise the scripts. Repeat this keystroke to turn them
off. You should be able to navigate the document and listen to audible
speech output, as well as reading the mathematical translation on the
braille display.
\end{enumerate}

\subsection{Emacs/emacspeak}
\label{subsubchap-emacs}

I assume you have downloaded or checked out the package from
subversion. If not, please see section~\ref{subchap-obtaining-source}

Installing emacs/emacspeak support for latex-access.

\begin{enumerate}
\item
\begin{itemize}
  \item A. Automated installation script.
  I have built an installation script titled ``setup.py'' found in the emacs
branch of this svn checkout.

Invoking with no arguments will install to \~/.emacs, otherwise if you know
what your doing and use an init file somewhere else, feel free to invoke
the script as follows:\\
\mbox{python setup.py <path-to-init-file>}

Most users who are happy with the default (\~/.emacs), may just invoke
the script as follows:\\
\mbox{python setup.py}

You may as well say yes to the prompt (y), to byte-compile
emacs-latex-access... This should improve the speed slightly. If it
fails or you do not want to byte compile the script can handle this
fine. (It modifies your init file according to whether it could
byte-compile or not.)

Continue following the prompts until the script has finished.
\item B. Manual installation. If you used the script, then skip to step
  2.
  Add the following to \~/.emacs. Change /path/to/svnroot to the actual
path to your svn local checkout. Eg. the directory containing the bulk
of the *.py files and the general readme.txt file.
Below lines should be appended to your .emacs.\\

; Emacs latex-access:\\
(setq latex-access-path "/path/to/svnroot")\\
(load (concat latex-access-path "/emacs/emacs-latex-access.el"))\\
(add-hook 'LaTeX-mode-hook 'latex-access-speech-on) ; comment this if you don't have emacspeak\\
(add-hook 'LaTeX-mode-hook 'latex-access-braille-on) ; comment if you don't have Braille \\
; End emacs Latex-access.\\

Note: Keep this structure if you wish to have the uninstall script work
with this installation.

See the comments if you don't have Braille or emacspeak present.

If you wish to use a byte-compiled file, for improved speed, replace .el
with .elc in the line:\\
(load (concat latex-access-path "/emacs/emacs-latex-access.el"))\\

To byte-compile the emacs-latex-access.el file, do:\\
emacs -batch -f batch-byte-compile emacs-latex-access.el\\
(From the emacs branch of the svn checkout)
\end{itemize}
\item 
Append to the PYTHONPATH environment variable the directory
/path/to/svnroot, replacing /path/to/svnroot with the actual path to
your svn checkout of latex-access if you used the manual
installation. Otherwise copy and paste the export path output by the
installation script.
This is usually done by an export line in .bash\_profile.
Ensure the variable is set before starting emacs.
\item Please install pymacs. 
On debian/ubuntu you may do:\\
\mbox{sudo apt-get install pymacs}\\
Otherwise follow the instructions provided at:
\url{http://pymacs.progiciels-bpi.ca/}
\item Restart emacs!
Now emacs should communicate correctly with latex-access.
\item If you wish to place any settings in your .emacs for latex-access,
place them under the line \\
; end latex-access\\
\begin{itemize}
\item A. To set how many linesabove the currently selected line should also be
translated add this line anywhere below\\
; end latex-access \\
(setq latex-access-linesabove 1)\\
This would Braille the current line and the one above, change 1 to
whatever setting you desire. 0 is just the current line.
\end{itemize}
\end{enumerate}

\subsection{NVDA}
\label{subsubchap-nvda}

\subsubsection{Requirements}

To use this plugin, you will need to install the following software/libraries:

\begin{itemize}
\item NVDA, version 2011.1 or later.  The NVDA website can be found at
  \url{http://www.nvda-project.org}, and the download page can be found 
  \href{http://www.nvda-project.org/download/}{here}.
\item The latest build of version
  2 of the python programming language.  Official website:
  \url{http://www.python.org/}.
\item The python for windows extentions, which
  can be downloaded at \url{http://sourceforge.net/projects/pywin32/}.  This
  package is required to registor the latex-access COM object and the
  matrix.
\item The latex-access scripts.  You probably already have these,
  but if not see section~\ref{subchap-obtaining-source}. (and be sure to remember
  which directory you checked out to)
\end{itemize}

\subsubsection{Registering the COM object and installing the globalPlugin.}

Assuming you have followed the steps above and installed everything you need, you can install the plugin and register the COM object by doing the following:

For registering the COM object:

\begin{itemize}
\item Type windows+r to go into the run dialog, then type cmd and press enter.
\item type ``cd'', then type in the location where you installed the latex-access scripts.  E.G.\\
\mbox{cd C:\textbackslash latex-access\textbackslash}
\item to register the COM object for latex-access translation, type:\\
\mbox{latex\_access\_COM.py}
\item To register the matrix, type:\\
\mbox{matrix\_processor.py}\\
  In both cases, you should hear iether the message ``latex\_access
  registered,'' or ``latex\_access\_matrix'' registered'' 
  which indicates that what you just did was successful.  If not,   then something is obviously wrong.  Remember that if you change the location of the latex-access scripts, you have to register the COM objects again.
\item Note: the object here is to run the specified python files with python, so the above will only work if python is the default program associated with .py files.  If it is not then try :\\
\mbox{python latex\_access\_COM.py}\\
however this only works if you have python in your path variable.  if this fails then use the full path to your python installation, for example :\\
\mbox{c:\textbackslash python26\textbackslash python.exe latex\_access\_COM.py}\\
then repeat with matrix\_processor.py.
\item type ``exit'' (without the quotes) in your currently open cmd shell.
\end{itemize}

\subsubsection{To install the plugin}

\begin{itemize}
\item Go to the nvda subdirectory in the latex-access folder, select latex\_access.py and copy it, either by pressing ctrl+C or right clicking on it and choosing ``copy.''
\item Now you need to paste this file in the appropriate folder so that the plugin will be recognised by NVDA.  Depending on your distribution/installation, the globalPlugins directory will be in one of the following places by default:
\begin{itemize}
\item Installed version of NVDA:
\begin{itemize}
\item For Windows XP:\\
\mbox{C:\textbackslash Documents and Settings\textbackslash username\textbackslash Application Data\textbackslash NVDA\textbackslash globalPlugins\textbackslash}
\item for Windows Vista/7:\\
\mbox{C:\textbackslash Users\textbackslash username\textbackslash AppData\textbackslash Roaming\textbackslash NVDA\textbackslash globalPlugins\textbackslash}
\end{itemize}
\item Portable archive or running from source:\\
\mbox{DirectoryOfArchive/distribution\textbackslash userConfig\textbackslash globalPlugins\textbackslash}
\end{itemize}
\item Once latex\_access.py is pasted, launch NVDA by what ever means possible, and if the error sound doesn't play, everything should go smoothly.
\end{itemize}

\subsubsection{A note about edit fields:}

If you use a  keystroke such as ctrl+m, it 
will only work if you are in an edit field, and one 
that has accessible text enabled at that. You 
will either get errors, or a different response from 
an application if you try to use the latex\_access global plugin 
in edit fields that are not native controls. I'm 
afraid that's just the way windows works. I know 
for a fact that notepad works with the plugin quite well, but 
in an editor such as EdSharp it does not.

\chapter{General usage}
\label{ch-general-usage}
\chapter{More advance usage}
\label{ch-more-advanced-usage}
\chapter{Extra components}
\label{ch-extra-components}
\chapter{For developers}
\label{ch-for-developers}

\appendix
\chapter{Jaws -- default key bindings}
\label{ap-jaws}
\chapter{NVDA -- default bindings}
\label{ap-nvda}
\chapter{Emacs -- functions and bindings}
\label{AP-emacs}
\chapter{Future features}
\label{AP-future}

\end{document}