\documentclass[12pt,a4paper]{report}
\usepackage{listings}
\usepackage{hyperref}

\title{The LaTeX-access manual}
\author{Alastair Irving <\href{mailto:alastair.irving@sjc.ox.ac.uk}{alastair.irving@sjc.ox.ac.uk}>\\
  Robin  Williams <\href{mailto:rmw205@exeter.ac.uk}{rmw205@exeter.ac.uk}>\\
  Daniel Dalton <\href{mailto:<daniel.dalton10@gmail.com}{daniel.dalton10@gmail.com}>\\
  Nathaniel Schmidt <\href{mailto:schmidty2244@gmail.com}{schmidty2244@gmail.com}>\\
  Stefan Moisei <\href{mailto:vortex37@gmail.com}{vortex37@gmail.com}>}
  \begin{document}
\maketitle

\tableofcontents

Latex-access -- provides a blind person with a more efficient means of
interacting with LaTeX for mathematical and scientific documents.

\chapter{Introduction}
\label{ch-introduction}

The latex-access project is designed to provide a realtime
translation of a line of LaTeX in to braille, concentrating
on the Nemeth and UEB codes, whichh can be read on a refreshable braille
display. This will greatly improve the ease of use of
LaTeX for many blind mathematicians and scientists. The project also
translates the current line into speech which is
easier to listen to than LaTeX source.

Note that this project is largely aimed at people wishing
to read LaTeX using a refreshable braille display and/or
speech synthesiser, and people who will probably
want to edit LaTeX documents. For example, as a
university student,
one may receive their worksheets in LaTeX format and
produce their work using LaTeX. Using the latex\_access
package, the student would be able to
get a fairly good translation of the question
and then an on-the-fly translation of their work as
they produce it. But if you are
not concerned with editing LaTeX documents
and simply want a braille translation of an
entire laTeX document, then this
project is not for you.

There is also a very low traffic mailing list, which is worth subscribing to if you have any
queries, problems, suggestions or ideas. All current developers are
subscribed to this list and are very willing to assist. To subscribe
send an email with the word ``subscribe'' in the subject to:
\href{mailto:latex-access-devel-request@lists.sourceforge.net}{latex-access-devel-request@lists.sourceforge.net}. To
post to the list send emails to \href{mailto:latex-access-devel@lists.sourceforge.net}{latex-access-devel@lists.sourceforge.net}

\section{Purpose}
\label{subchap-purpose}
It is widely thought that LaTeX is a good system for a blind
mathematician or scientist to use to create and read
scientific documents, as it is a linear code and so the user does
not have to perceive two-dimensional concepts, such
as fractions and column vectors.
By reading this linear code, a blind person can take in and
understand scientific documents in the same way that a
sighted person would do by studying a printed document.
It should be noted that normally, laTeX is just a source from which
documents are converted in to an
attractive-looking, typeset document that can be printed or viewed
on screen, often in a .pdf, .dvi or .ps format. For
various technical reasons, documents in such formats are
currently inaccessible with current screen-reading technology.
The best current solution therefore is not to concern
ourselves with documents in these formats, but rather to
read
and interpret the LaTeX source code itself.

\section{Reading a LaTeX document}
\label{subchap-reading-latex-document}

It is entirely possible to read a LaTeX document simply by reading
the LaTeX source itself. This however, is often a
time-consuming and pain-staking process, and it is often not
particularly nice to read. For example, the LaTeX source
for the quadratic formula is\\
\begin{lstlisting}[language = tex]
$$x=\frac{-b\pm\sqrt{b^2-4ac}}{2a}$$
\end{lstlisting}

It is therefore the aim of the project to translate a
line of LaTeX in to a line of Niemeth braille code, which
can be
read using a refreshable braille display. The project
also provides an audible translation of the
LaTeX source
which can be output through current screen-reading
technology.          

\section{Current features}
\label{subchap-current-features}

latex-access currently contains the following features.

\begin{itemize}
\item Translation of several mathematical expressions from LaTeX to
Niemeth braille. These include, but are not confined
to:
\begin{itemize}
\item Translation of fractions, both numerical and
algebraic.
\item Translation of trigonometric
functions and hyperbolic functions.
\item Translation of powers,
including square roots.
\item Translation of
expressions used in calculus, including partial derivatives.
\item Translation of two component and three component column vectors,
  not in to Niemeth braille format but in to a row vector so that they
  can be read on a single line braille display.
\item Translation of several mathematical symbols, such as the Greek letters.
\item Many commands used to create a visually attractive document are
  either translated or ignored, often it is not necessary to see some
  formatting commands.
\end{itemize}

\item Translation of several of the above to audible speech.
\item A matrix browser feature to enable easier reading of larger
  matrices in LaTeX, see the description below.
\item Support for custom defined LaTeX commands.
\end{itemize}

\chapter{Installation}
\label{ch-installation}

\section{Obtaining the source}
\label{subchap-obtaining-source}
The package is hosted by Git at,
\url{https://github.com/latex-access/latex-access}.

This link will take you to a web interface of the git repository tree, but you'll
probably want to clone (checkout) the code so you can install it. If you run
windows see section~\ref{subsubchap-windows}, and if you run linux
see section~\ref{subsubchap-linux}.

\subsection{Linux}
\label{subsubchap-linux}

Under Linux, the standard git command-line client works
well, or you can use Subversion to checkout the Git repository. One of these can usually be installed on debian based distros by running
\begin{verbatim}
apt-get install git
\end{verbatim}
or
\begin{verbatim}
apt-get install subversion
\end{verbatim}
\# (as root)\\
Then type (whether using Git or svn respectively)
\begin{verbatim}
git clone https://github.com/latex-access/latex-access.git
\end{verbatim}
or
\begin{verbatim}
svn co https://github.com/latex-access/latex-access latex-access
\end{verbatim}

This will check the package out into the directory latex-access.

In future feel free to run
\begin{verbatim}
git pull
\end{verbatim}
or
\begin{verbatim}
svn up
\end{verbatim}
(from within the latex-access directory) to pull the latest updates from
the server.

\subsection{Windows}
\label{subsubchap-windows}

It is recommended to use Git for Windows,, which you can download for the command line at the following link:\\
\url{https://git-scm.com/download/win}

On windows, if you prefer a graphical user interface (GUI), you also have the potential compromise of using the Tortoise SVN client.\\
\url{http://tortoisesvn.tigris.org/}

Once you have checked out the source code from\\
\url{https://github.com/latex-access/latex-access.git}\\
continue with the installation
process.

Note, you should periodically pull the latest updates from the server to
get the latest and greatest features of latex-access. You can do this by simply typing
\begin{verbatim}
git pull
\end{verbatim}
within the directory you cloned the Latex-access repository to.

\section{Installing the package}

Currently latex-access interacts with three different
programs. Therefore, the installation process varies slightly.

\begin{description}
\item [Jaws for Windows] most developed front-end to the project, and
  allows Jaws to interact with latex-access within most editors.
\item [Emacs/Emacspeak] Provides both Braille (via BRLTTY) and speech access when
    working with latex under the emacs editor. This means anyone using a
    Linux system can use the project. Usage under Linux is very stable,
    but there are still a few little things to be done, see the README
    in the emacs sub-directory for details. Currently has not been tested with
    the Windows emacs version.
\item [NVDA] provides latex-access for usage with most
      editors. Under heavy development still, but it provides many
      features of latex-access. See section~\ref{subsubchap-nvda} for
      installation instructions.
\end{description}

For specific notes on each individual front-end it is recommended looking at the
READMEs in their respective sub-directories, as these are much more likely to be up-to-date. You can also find details about
contacting the developer of the specific front-end in question from the readme.

\subsection{Emacs/emacspeak}
\label{subsubchap-emacs}

I assume you have downloaded or checked out the package from
subversion. If not, please see section~\ref{subchap-obtaining-source}

Installing emacs/emacspeak support for latex-access.

\begin{enumerate}
\item
\begin{itemize}
  \item A. Automated installation script.
  I have built an installation script titled ``setup.py'' found in the emacs
branch of this svn checkout.

Invoking with no arguments will install to
\begin{verbatim}
~/.emacs
\end{verbatim}, otherwise if you know
what your doing and use an init file somewhere else, feel free to invoke
the script as follows:
\begin{verbatim}
python setup.py <path-to-init-file>
\end{verbatim}

Most users who are happy with the default (
\begin{verbatim}
~/.emacs
\end{verbatim}
), may just invoke
the script as follows:
\begin{verbatim}
python setup.py
\end{verbatim}

You may as well say yes to the prompt (y), to byte-compile
emacs-latex-access... This should improve the speed slightly. If it
fails or you do not want to byte compile the script can handle this
fine. (It modifies your init file according to whether it could
byte-compile or not.)

Continue following the prompts until the script has finished.
\item B. Manual installation. If you used the script, then skip to step
  2.
  Add the following to
  \begin{verbatim}
~/.emacs
\end{verbatim}. Change /path/to/svnroot to the actual
path to your svn local checkout. Eg. the directory containing the bulk
of the *.py files and the general readme.txt file.
Below lines should be appended to your .emacs.\\

; Emacs latex-access:\\
(setq latex-access-path "/path/to/svnroot")\\
(load (concat latex-access-path "/emacs/emacs-latex-access.el"))\\
(add-hook 'LaTeX-mode-hook 'latex-access-speech-on) ; comment this if you don't have emacspeak\\
(add-hook 'LaTeX-mode-hook 'latex-access-braille-on) ; comment if you don't have Braille \\
; End emacs Latex-access.\\

Note: Keep this structure if you wish to have the uninstall script work
with this installation.

See the comments if you don't have Braille or emacspeak present.

If you wish to use a byte-compiled file, for improved speed, replace .el
with .elc in the line:\\
(load (concat latex-access-path "/emacs/emacs-latex-access.el"))\\

To byte-compile the emacs-latex-access.el file, do:\\
emacs -batch -f batch-byte-compile emacs-latex-access.el\\
(From the emacs branch of the svn checkout)
\end{itemize}
\item 
Append to the PYTHONPATH environment variable the directory
/path/to/svnroot, replacing /path/to/svnroot with the actual path to
your svn checkout of latex-access if you used the manual
installation. Otherwise copy and paste the export path output by the
installation script.
This is usually done by an export line in .bash\_profile.
Ensure the variable is set before starting emacs.
\item Please install pymacs. 
On debian/ubuntu you may do:
\begin{verbatim}
sudo apt-get install pymacs
\end{verbatim}
Otherwise follow the instructions provided at:
\url{http://pymacs.progiciels-bpi.ca/}
\item Restart emacs!
Now emacs should communicate correctly with latex-access.
\item If you wish to place any settings in your .emacs for latex-access,
place them under the line \\
; end latex-access\\
\begin{itemize}
\item A. To set how many linesabove the currently selected line should also be
translated add this line anywhere below\\
; end latex-access \\
(setq latex-access-linesabove 1)\\
This would Braille the current line and the one above, change 1 to
whatever setting you desire. 0 is just the current line.
\end{itemize}
\end{enumerate}

\chapter{For developers}
\label{ch-for-devs}

\end{document}